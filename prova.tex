\documentclass[12pt, a4paper,twoside]{tesi_upf}

%CODIFICACIÓ
%\usepackage[latin1]{inputenc}

%DIAGRAMA DE GANTT
\usepackage{pgfgantt}

%IDIOMES
\usepackage[catalan,english]{babel}

%PER A INCLOURE GRÀFICS I EL LOGO DE LA UPF
\usepackage{graphicx}
\usepackage{caption}
\usepackage{acronym}
\usepackage{multirow}
%FONTS TIMES O GARAMOND, 
\usepackage{times}
%\usepackage{garamond}
\usepackage{url}

\usepackage{pdfpages}
%SENSE HEADINGS: NO MODIFICAR
\pagestyle{plain}

%PER A L'ÍNDEX DE MATÈRIES
\usepackage{makeidx}
\makeindex

%ESTIL DE BIBLIOGRAFIA
%\bibliographystyle{apalike}

%AQUEST DOCUMENT ÉS EN ANGLÈS
\selectlanguage{english}

%AFEGIU EN AQUESTA PART LES VOSTRES DADES
\title{Aerial sensor platform}
\author{Gonzalo Hermida Ruiz}
\thyear{2014}
\department{Departament de Tecnologies de la Informació i les Comunicacions (DTIC)}
\supervisor{Jaume Barceló, Carles Ayesa, Luis Sanabria}

\begin{document}

\frontmatter

\maketitle

\cleardoublepage

%%%%%% Dedicatòria; si no es vol posar, comenteu fins a final de dedicatòria

\noindent To my family.

\cleardoublepage

%%%%%% Final de dedicatòria

%%%%%% Agraïments; si no es vol posar, comenteu fins a final de agraïments
\noindent {\Large \sffamily Acknowledgments}
\\[12pt] 
        
I would like to thank to 
\\[8pt]

Also, to 
\\[8pt]

At last, 

\cleardoublepage

%%%%%% Final dels agraïments

%ABSTRACT EN DOS IDIOMES. COM A MÍNIM CATALÀ. SI L'ALTRE ÉS EN CASTELLA CANVIEU EL QUE POSA ABSTRACT
\selectlanguage{english}
\section*{\Large \sffamily Abstract}

This  work  addresses  the  gathering  of  data  in  areas  of  difficult  access  or  which  are potentially dangerous. Examples include high tension power lines, collapsed buildings and fire areas. We build a flying platform with the ability of carrying light sensors (e.g., small cameras or infrared cameras) and transmit the sensed data wirelessly to a control point. The platform is a highly manoeuvrable multicopter that uses the Arduino microcontroller and the multiwii software.

\selectlanguage{catalan}
\vspace*{\fill}
\section*{\Large \sffamily  Resum}

Abtracte en català

\vspace*{\fill}

\selectlanguage{english}
\cleardoublepage
%FIN DE ABSTRACTE

%PREFACI OPCIONAL. SI NO ES VOL, COMENTEU FINS EL FINAL DE PREFACI
%{\bf Prefaci}
%
%\cleardoublepage
%FINAL DE PREFACI


%TAULA DE CONTINGUTS: OBLIGATÒRIA
\tableofcontents

%INDEX DE FIGURES; NOMÉS ES POSA SI HI HA FIGURES
\listoffigures
%Fa que aparegui al sumari
\addcontentsline{toc}{chapter}{List of figures}

%INDEX DE TAULES; NOMÉS ES POSA SI HI HA TAULES
\listoftables
%Fa que aparegui al sumari
\addcontentsline{toc}{chapter}{List of tables}

%INDEX DE ACRONIMS
\cleardoublepage
\thispagestyle{empty}
\addcontentsline{toc}{chapter}{List of Abbreviations}
\vspace*{1.95cm} \hspace*{-0.155cm} %,88
\textbf{{\huge \sffamily List of Abbreviations}\\}
\vspace*{0.5cm}         
\begin{acronym}
%\acro{AP}{Access Point}
%\acro{BMX6}{BatMan-eXperimental version 6}
%\acro{BSS}{Basic Service Set}
\end{acronym}

%COMENÇA EL TEXT
\mainmatter
\chapter{Introduction}


%\\[12pt]


%\\[12pt] 


%\\[12pt]


%\\[12pt]


%\\[12pt]

The streaming solution we propose, perfectly fits those networks and specially a scenario with the mobile node where we may want to share the content we are watching at that moment.  

%\chapter{State of the Art}

\chapter{Planning Report}

The following sections explain the tasks that I will do in the course of this project.

\section{Pieces adquisition}

This item includes the estimate time to plan which pieces are needed, how many of each, the purchase of them and the average waiting time until them arribe. 

\section{Assembling infraestructure device}

This item includes the required time to assembling the device once the pieces have arribed and we have all the needed tools.

\section{Software Implementation}

This item includes the required time to install the different software on the arduinos: the transmissor, the receptor and the controller; plus all the required software to be able to configure the arduinos through the PC.

\section{Flight Tests}

This item includes the required time to do the flight tests itself and the time to calibrate the device based on the results obtained on the tests and their interpretation.

\section{Camera incorporation}

This item includes the time needed to incorporate a camera to the device in order to take video images and transmitt it on live.

\section{Device improvements}

This item includes the required time to incorporate a bluetooth module to facilitate the connection between the arduino and the PC on a wireless mode, plus the incorporation of a GPS module, in order to extend the device possibilities.

\section{Final report}

The wording of the report is performed in parallel with the tasks that are being performed.

\section{Gantt chart}

\begin{figure}[ftbp]
  \begin{center}

    \begin{ganttchart}
    [y unit title=0.4cm, 
    y unit chart=0.5cm,
    vgrid,hgrid,
    title label anchor/.style={below=-1.6ex},
    title left shift=.05,
    title right shift=-.05,
    title height=1,
    bar/.style={fill=gray!50},
    incomplete/.style={fill=white},
    progress label text={},
    bar height=0.7,
    group right shift=0,
    group top shift=.6,
    group height=.3,
    group peaks height={}{}{.1}]{1}{24}
    
    %labels
    \gantttitle{2013-2014}{24} \\
    \gantttitle{October}{3}
    \gantttitle{November}{3} 
    \gantttitle{December}{3}
    \gantttitle{January}{3} 
    \gantttitle{February}{3} 
    \gantttitle{March}{3} 
    \gantttitle{April}{3} 
    \gantttitle{May}{3}\\
    
    %tasks
    \ganttbar{Task 1}{2}{6} \\
    \ganttbar{Task 2}{7}{12} \\
    \ganttbar{Task 3}{10}{15} \\
    \ganttbar{Task 4}{16}{24} \\
    \ganttbar{Task 5}{19}{24} \\
    \ganttbar{Task 6}{22}{24} \\
    \ganttbar{Final report}{2}{24}

    %relations 
    \ganttlink{elem0}{elem1}  
    \ganttlink{elem2}{elem3} 


   \end{ganttchart}
  \end{center}
\end{figure}




\end{document}
